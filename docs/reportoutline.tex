
\documentclass[]{article}

%opening


\title{Project Report Outline}
\author{}
\date{}
\begin{document}
\maketitle

\section{Introduction}
\label{sec:introduction}

\subsection{Basics}
\label{sub:basics}

Discussion of basics of carbon based life, and definition of metabolic networks.

Metabolic networks very similar use only a small set of reactions and compounds $\rightarrow$ possible explanations, evolution.

What my project is about: explain why this set of reactions was chosen, why the uniformity? Test if reason is evolution.

\subsection{Biological evolution}
\label{sub:About evolution}
Three main ingredients: reproduction, mutation selection, why is it important for physics, maths and computer science?

\subsection{Metabolic networks as graphs}
\label{sub:About Graph THeory}

Topological similarities to human social networks, internet, small worldliness, error tolerance, modularity. 

\subsection{Goals of the project}
\label{sub:What I will do}

Subset of molecules used, simulating population of artificial organisms with selection process like evolution. Looking at patterns of improvements.

\subsection{Connection to physics}
\label{sub:Where is the physics}

Free energy change calculations, survival of the fittest using the moran process, Metropolis algorithm? Enthropy and Inverse participation ratio as a measure of disorder. 


\section{Methods}
\label{sec:methods}

\subsection{The data provided}
\label{sub:What was I provided with?}
List of reactants and reactions, with standard free E changes. 
\subsubsection{Restrictions on chemistry}
\label{ssub:Restrictions on chemistry}
What compounds are used, why?
Previous author's restrictions on artificial chemistries. Should this go to Intro?
\subsection{Storing the reaction network}
\label{sub:Storing the reaction network}
Structure of the input files -$\rightarrow$ algorithm to read them in $\rightarrow$ store them. 

The neighbour list and it's use in adding reactions.
\subsection{Calculating the free energy change}
\label{sub:The free energy change}
The physics of the formula to calculate it. The role of deciding the direction of reactions.  $\rightarrow$ valid ranges  that can be changed at the beginning of the program.

\subsection{Random number generation}
\label{sub:Random number generation}

Good quality random numbers $\rightarrow$ mt19937 Mersenne Twister. Reproducibility is important $\rightarrow$ seeds always known. 



\subsection{Flux Balance analysis}
\label{sub:Flux Balance analysis}

Detailed description $\rightarrow$ equations, example matrix? Which fluxes are free? Which fluxes are limited? GLPK 


\subsection{In silico Evolution}
\label{sub:Implementing evolution}
\subsubsection{Mutations}
\label{ssub:Mutations}
The possible mutations, why it makes sense to use only these, and possible other mutations that could have been used. The rate of mutations. 
\subsubsection{Reproduction}
\label{ssub:Reproduction}
survival of the fittest using the Moran process - Metropolis algorithm?

\subsubsection{Competition}
\label{ssub:Competition}
Goal functions? $\rightarrow$ changing goal functions? Talk about trial networks (ethanol producing)?

\subsection{Visualization of metabolic pathways}
\label{sub:visualization}

the use of Cytoscape $\rightarrow$ xgmml format and the  program part to output that

\subsection{Population simulation}
\label{sub:population_simulation}
How to store the population, how to choose which one to mutate (as in \ref{ssub:Reproduction}. 

New method of storing indexes only. Include the previous method too?

\subsubsection{Parallel simulations}
\label{ssub:Paralell simulations}
Multiple populations with same initial network  run paralell$\rightarrow$ different random seed. Cplab machines, talk about the distributing/collecting scripts?


\subsection{Comparing populations}
\label{sub:comparing_populations}


\subsubsection{Similarity Index}
\label{ssub:Similarity Index}

How to calculate, why is it useful? Can be calculated for nonzero flux reactions or all available ones too. 

\subsubsection{Inverse participation ratio}
\label{ssub:Inverse participation ratio}
IPR vs Enthropy

\section{Results}
\label{sec:results}

\subsection{Similar networks reached in paralell runs}
\label{sub:similar_networks_reached_in_paralell_runs}


If mutations happen with 100$\%$ then rather fast improvement to best network. Very diverse populations in terms of available reactions, but somewhat similar within a single run. Different runs very different available reactions. For the nonzero flux reactions within a run is almost completely homogeneous, and lots of similarity between runs too. 

If the mutation probabilities decrease improvements become more difficult.

\subsection{Limitations}
\label{sub:limitations}
Dynamics of chemical reactions ignored

Restrictions in the types of mutations

Small interactions between cells

Too simple goal function



\section{Discussion}
\label{sec:discussion}




\end{document}
